% Options for packages loaded elsewhere
\PassOptionsToPackage{unicode}{hyperref}
\PassOptionsToPackage{hyphens}{url}
%
\documentclass[
]{article}
\usepackage{amsmath,amssymb}
\usepackage{iftex}
\ifPDFTeX
  \usepackage[T1]{fontenc}
  \usepackage[utf8]{inputenc}
  \usepackage{textcomp} % provide euro and other symbols
\else % if luatex or xetex
  \usepackage{unicode-math} % this also loads fontspec
  \defaultfontfeatures{Scale=MatchLowercase}
  \defaultfontfeatures[\rmfamily]{Ligatures=TeX,Scale=1}
\fi
\usepackage{lmodern}
\ifPDFTeX\else
  % xetex/luatex font selection
\fi
% Use upquote if available, for straight quotes in verbatim environments
\IfFileExists{upquote.sty}{\usepackage{upquote}}{}
\IfFileExists{microtype.sty}{% use microtype if available
  \usepackage[]{microtype}
  \UseMicrotypeSet[protrusion]{basicmath} % disable protrusion for tt fonts
}{}
\makeatletter
\@ifundefined{KOMAClassName}{% if non-KOMA class
  \IfFileExists{parskip.sty}{%
    \usepackage{parskip}
  }{% else
    \setlength{\parindent}{0pt}
    \setlength{\parskip}{6pt plus 2pt minus 1pt}}
}{% if KOMA class
  \KOMAoptions{parskip=half}}
\makeatother
\usepackage{xcolor}
\usepackage[margin=1in]{geometry}
\usepackage{color}
\usepackage{fancyvrb}
\newcommand{\VerbBar}{|}
\newcommand{\VERB}{\Verb[commandchars=\\\{\}]}
\DefineVerbatimEnvironment{Highlighting}{Verbatim}{commandchars=\\\{\}}
% Add ',fontsize=\small' for more characters per line
\usepackage{framed}
\definecolor{shadecolor}{RGB}{248,248,248}
\newenvironment{Shaded}{\begin{snugshade}}{\end{snugshade}}
\newcommand{\AlertTok}[1]{\textcolor[rgb]{0.94,0.16,0.16}{#1}}
\newcommand{\AnnotationTok}[1]{\textcolor[rgb]{0.56,0.35,0.01}{\textbf{\textit{#1}}}}
\newcommand{\AttributeTok}[1]{\textcolor[rgb]{0.13,0.29,0.53}{#1}}
\newcommand{\BaseNTok}[1]{\textcolor[rgb]{0.00,0.00,0.81}{#1}}
\newcommand{\BuiltInTok}[1]{#1}
\newcommand{\CharTok}[1]{\textcolor[rgb]{0.31,0.60,0.02}{#1}}
\newcommand{\CommentTok}[1]{\textcolor[rgb]{0.56,0.35,0.01}{\textit{#1}}}
\newcommand{\CommentVarTok}[1]{\textcolor[rgb]{0.56,0.35,0.01}{\textbf{\textit{#1}}}}
\newcommand{\ConstantTok}[1]{\textcolor[rgb]{0.56,0.35,0.01}{#1}}
\newcommand{\ControlFlowTok}[1]{\textcolor[rgb]{0.13,0.29,0.53}{\textbf{#1}}}
\newcommand{\DataTypeTok}[1]{\textcolor[rgb]{0.13,0.29,0.53}{#1}}
\newcommand{\DecValTok}[1]{\textcolor[rgb]{0.00,0.00,0.81}{#1}}
\newcommand{\DocumentationTok}[1]{\textcolor[rgb]{0.56,0.35,0.01}{\textbf{\textit{#1}}}}
\newcommand{\ErrorTok}[1]{\textcolor[rgb]{0.64,0.00,0.00}{\textbf{#1}}}
\newcommand{\ExtensionTok}[1]{#1}
\newcommand{\FloatTok}[1]{\textcolor[rgb]{0.00,0.00,0.81}{#1}}
\newcommand{\FunctionTok}[1]{\textcolor[rgb]{0.13,0.29,0.53}{\textbf{#1}}}
\newcommand{\ImportTok}[1]{#1}
\newcommand{\InformationTok}[1]{\textcolor[rgb]{0.56,0.35,0.01}{\textbf{\textit{#1}}}}
\newcommand{\KeywordTok}[1]{\textcolor[rgb]{0.13,0.29,0.53}{\textbf{#1}}}
\newcommand{\NormalTok}[1]{#1}
\newcommand{\OperatorTok}[1]{\textcolor[rgb]{0.81,0.36,0.00}{\textbf{#1}}}
\newcommand{\OtherTok}[1]{\textcolor[rgb]{0.56,0.35,0.01}{#1}}
\newcommand{\PreprocessorTok}[1]{\textcolor[rgb]{0.56,0.35,0.01}{\textit{#1}}}
\newcommand{\RegionMarkerTok}[1]{#1}
\newcommand{\SpecialCharTok}[1]{\textcolor[rgb]{0.81,0.36,0.00}{\textbf{#1}}}
\newcommand{\SpecialStringTok}[1]{\textcolor[rgb]{0.31,0.60,0.02}{#1}}
\newcommand{\StringTok}[1]{\textcolor[rgb]{0.31,0.60,0.02}{#1}}
\newcommand{\VariableTok}[1]{\textcolor[rgb]{0.00,0.00,0.00}{#1}}
\newcommand{\VerbatimStringTok}[1]{\textcolor[rgb]{0.31,0.60,0.02}{#1}}
\newcommand{\WarningTok}[1]{\textcolor[rgb]{0.56,0.35,0.01}{\textbf{\textit{#1}}}}
\usepackage{graphicx}
\makeatletter
\newsavebox\pandoc@box
\newcommand*\pandocbounded[1]{% scales image to fit in text height/width
  \sbox\pandoc@box{#1}%
  \Gscale@div\@tempa{\textheight}{\dimexpr\ht\pandoc@box+\dp\pandoc@box\relax}%
  \Gscale@div\@tempb{\linewidth}{\wd\pandoc@box}%
  \ifdim\@tempb\p@<\@tempa\p@\let\@tempa\@tempb\fi% select the smaller of both
  \ifdim\@tempa\p@<\p@\scalebox{\@tempa}{\usebox\pandoc@box}%
  \else\usebox{\pandoc@box}%
  \fi%
}
% Set default figure placement to htbp
\def\fps@figure{htbp}
\makeatother
\setlength{\emergencystretch}{3em} % prevent overfull lines
\providecommand{\tightlist}{%
  \setlength{\itemsep}{0pt}\setlength{\parskip}{0pt}}
\setcounter{secnumdepth}{-\maxdimen} % remove section numbering
\usepackage{bookmark}
\IfFileExists{xurl.sty}{\usepackage{xurl}}{} % add URL line breaks if available
\urlstyle{same}
\hypersetup{
  pdftitle={Chapter 1 - Importing and Visualizing Functional Data},
  pdfauthor={VP},
  hidelinks,
  pdfcreator={LaTeX via pandoc}}

\title{Chapter 1 - Importing and Visualizing Functional Data}
\author{VP}
\date{2025-08-15}

\begin{document}
\maketitle

\section{COVID-19 US Mortality Data}\label{covid-19-us-mortality-data}

\paragraph{Weekly all-cause mortality
data}\label{weekly-all-cause-mortality-data}

We assemble exploratory visualizations for all-cause and COVID-19 weekly
mortality data in the US. The all‑cause mortality data spans 222 weeks,
from the week ending January 14, 2017 through the week ending April 10,
2021.The Data is sourced from National Center for Health Statistics.The
data can be accessed via the REFUND package in R.

The COVID‑19 mortality data consist of the weekly mortality, where
COVID‑19 is listed as the underlying cause of death. This spans for 68
weeks, from the week ending January 4, 2020 to the week ending April 17,
2021, and is sourced from National Center for Health Statistics.The data
can be accessed via the REFUND package in R.

We aim to apply functional data analysis methods, implemented via the
tidyfun package, to visualize U.S. mortality trends with a focus on the
year 2020. The analysis addresses two objectives:

\begin{enumerate}
\def\labelenumi{\arabic{enumi}.}
\item
  To visualize all-cause total and excess mortality(mortality in a week
  in 2020 minus mortality in the coresponding week in 2019)
\item
  To visualize the temporal pattern of the total COVID-19 mortality.
\end{enumerate}

\begin{Shaded}
\begin{Highlighting}[]
\CommentTok{\# ─Packages───────────────────}
\CommentTok{\# List of required packages}
\NormalTok{packages }\OtherTok{\textless{}{-}} \FunctionTok{c}\NormalTok{(}
  \StringTok{"refund"}\NormalTok{,}
  \StringTok{"tidyverse"}\NormalTok{,}
  \StringTok{"tidyfun"}
\NormalTok{)}

\CommentTok{\# Function to install missing packages}
\NormalTok{install\_if\_missing }\OtherTok{\textless{}{-}} \ControlFlowTok{function}\NormalTok{(pkg) \{}
  \ControlFlowTok{if}\NormalTok{ (}\SpecialCharTok{!}\FunctionTok{requireNamespace}\NormalTok{(pkg, }\AttributeTok{quietly =} \ConstantTok{TRUE}\NormalTok{)) \{}
    \FunctionTok{install.packages}\NormalTok{(pkg)}
\NormalTok{  \}}
\NormalTok{\}}

\CommentTok{\# Install missing packages}
\FunctionTok{invisible}\NormalTok{(}\FunctionTok{lapply}\NormalTok{(packages, install\_if\_missing))}



\CommentTok{\#  calling all the packages }
\FunctionTok{library}\NormalTok{(refund)      }
\FunctionTok{library}\NormalTok{(tidyverse)   }
\FunctionTok{library}\NormalTok{(tidyfun)     }

\FunctionTok{set.seed}\NormalTok{(}\DecValTok{53615}\NormalTok{)}
\end{Highlighting}
\end{Shaded}

\begin{Shaded}
\begin{Highlighting}[]
\CommentTok{\# ────Overview───────────────────}
\FunctionTok{data}\NormalTok{(}\StringTok{"COVID19"}\NormalTok{, }\AttributeTok{package =} \StringTok{"refund"}\NormalTok{) }
\FunctionTok{names}\NormalTok{(COVID19)}
\end{Highlighting}
\end{Shaded}

\begin{verbatim}
##  [1] "US_weekly_mort"                      "US_weekly_mort_dates"               
##  [3] "US_weekly_mort_CV19"                 "US_weekly_mort_CV19_dates"          
##  [5] "US_weekly_excess_mort_2020"          "US_weekly_excess_mort_2020_dates"   
##  [7] "US_states_names"                     "US_states_population"               
##  [9] "States_excess_mortality"             "States_excess_mortality_per_million"
## [11] "States_CV19_mortality"               "States_CV19_mortality_per_million"
\end{verbatim}

\begin{Shaded}
\begin{Highlighting}[]
\CommentTok{\# ─────────────helpers───────────────────}
\CommentTok{\#Conversion of the dates into  numbers (days since the very first week)}
\CommentTok{\#because the args of tfd  need numbers, not dates.}
\NormalTok{num\_grid }\OtherTok{\textless{}{-}} \ControlFlowTok{function}\NormalTok{(dates, }\AttributeTok{ref =} \FunctionTok{min}\NormalTok{(dates)) }\FunctionTok{as.numeric}\NormalTok{(dates }\SpecialCharTok{{-}}\NormalTok{ ref)}
\NormalTok{reference\_date }\OtherTok{\textless{}{-}} \FunctionTok{as.Date}\NormalTok{(}\StringTok{"2020{-}01{-}01"}\NormalTok{)}
\end{Highlighting}
\end{Shaded}

\begin{Shaded}
\begin{Highlighting}[]
\CommentTok{\#Data}
\NormalTok{counts\_state }\OtherTok{\textless{}{-}}\NormalTok{ COVID19}\SpecialCharTok{$}\NormalTok{US\_weekly\_mort}
\NormalTok{date\_state }\OtherTok{\textless{}{-}}\NormalTok{ COVID19}\SpecialCharTok{$}\NormalTok{US\_weekly\_mort\_dates}
\end{Highlighting}
\end{Shaded}

\begin{Shaded}
\begin{Highlighting}[]
\CommentTok{\# ───Data Transformation ─────────────}
\CommentTok{\#Its standard in tidyfun to create a tibble(part of the pipeline)}
\NormalTok{nat\_weekly }\OtherTok{\textless{}{-}} \FunctionTok{tibble}\NormalTok{(}
  \CommentTok{\# week{-}start dates}
  \AttributeTok{date   =}\NormalTok{ date\_state,}
  \CommentTok{\#the weekly all{-}cause death counts  }
  \CommentTok{\#Divide by 1000 to indicate numbers in thousands}
  \AttributeTok{deaths =}\NormalTok{ counts\_state }\SpecialCharTok{/} \DecValTok{1000}       
\NormalTok{)}


\CommentTok{\# Shading rectangles: actual full weeks of 2019 and 2020}
\CommentTok{\#The red area is 2020 and the blue area is 2019}

\NormalTok{shade }\OtherTok{\textless{}{-}}\NormalTok{ nat\_weekly }\SpecialCharTok{\%\textgreater{}\%}
  \FunctionTok{filter}\NormalTok{(}\FunctionTok{year}\NormalTok{(date) }\SpecialCharTok{\%in\%} \FunctionTok{c}\NormalTok{(}\DecValTok{2019}\NormalTok{, }\DecValTok{2020}\NormalTok{)) }\SpecialCharTok{\%\textgreater{}\%}
  \FunctionTok{mutate}\NormalTok{(}
    \AttributeTok{year =} \FunctionTok{year}\NormalTok{(date),}
    \AttributeTok{day\_num =} \FunctionTok{num\_grid}\NormalTok{(date, }\AttributeTok{ref =} \FunctionTok{min}\NormalTok{(nat\_weekly}\SpecialCharTok{$}\NormalTok{date))}
\NormalTok{  ) }\SpecialCharTok{\%\textgreater{}\%}
  \FunctionTok{group\_by}\NormalTok{(year) }\SpecialCharTok{\%\textgreater{}\%}
  \FunctionTok{summarise}\NormalTok{(}
    \CommentTok{\#For each year, get the min and max week{-}index}
    \AttributeTok{xmin =} \FunctionTok{min}\NormalTok{(day\_num),}
    \AttributeTok{xmax =} \FunctionTok{max}\NormalTok{(day\_num),}
    \AttributeTok{.groups =} \StringTok{"drop"}
\NormalTok{  ) }\SpecialCharTok{\%\textgreater{}\%}
  \FunctionTok{mutate}\NormalTok{(}
    \AttributeTok{band =} \FunctionTok{as.character}\NormalTok{(year),}
    \AttributeTok{colour =} \FunctionTok{c}\NormalTok{(}\StringTok{"blue"}\NormalTok{, }\StringTok{"red"}\NormalTok{)}
\NormalTok{  )}

\CommentTok{\# X{-}axis year labels at first week of each year}
\NormalTok{year\_ticks }\OtherTok{\textless{}{-}}\NormalTok{ nat\_weekly }\SpecialCharTok{\%\textgreater{}\%}
\CommentTok{\#we  add two new columns}
  \FunctionTok{mutate}\NormalTok{(}
    \AttributeTok{year =} \FunctionTok{year}\NormalTok{(date),}
    \AttributeTok{day\_num =} \FunctionTok{num\_grid}\NormalTok{(date)}
\NormalTok{  ) }\SpecialCharTok{\%\textgreater{}\%}
  \FunctionTok{group\_by}\NormalTok{(year) }\SpecialCharTok{\%\textgreater{}\%}
  \CommentTok{\#within each year, we take the row with the smallest date for the plot}
  \FunctionTok{slice\_min}\NormalTok{(date, }\AttributeTok{n =} \DecValTok{1}\NormalTok{) }\SpecialCharTok{\%\textgreater{}\%}
  \FunctionTok{ungroup}\NormalTok{()}



\CommentTok{\#  Plot weekly all{-}cause US mortality 2017 to 2020}
\CommentTok{\#. The reason for the gap is because if you run this:}
\CommentTok{\# print(nat\_weekly \%\textgreater{}\%}
\CommentTok{\#         filter(year(date) \%in\% c(2019, 2020)),n =104)}
\CommentTok{\#You will see that there is a week between the last date of 2019 and the first date of 2020}
\end{Highlighting}
\end{Shaded}

\begin{Shaded}
\begin{Highlighting}[]
\CommentTok{\# ─────────────────── Turn into tfd ─────────────}
\NormalTok{nat\_weekly\_tfd }\OtherTok{\textless{}{-}} \FunctionTok{tfd}\NormalTok{(}
  \CommentTok{\#We reshape the deaths into a 1{-}row matrix so it’s literally one curve through 2017 to 2020.}
  \FunctionTok{matrix}\NormalTok{(nat\_weekly}\SpecialCharTok{$}\NormalTok{deaths, }\AttributeTok{nrow =} \DecValTok{1}\NormalTok{),}
  \AttributeTok{arg =} \FunctionTok{num\_grid}\NormalTok{(nat\_weekly}\SpecialCharTok{$}\NormalTok{date)}
\NormalTok{)}
\NormalTok{nat\_weekly\_tfd}
\end{Highlighting}
\end{Shaded}

\begin{verbatim}
## tfd[1] on (0,1442) based on 207 evaluations each
## interpolation by tf_approx_linear 
## [1]: ( 0,61);( 7,59);(14,58); ...
\end{verbatim}

\begin{Shaded}
\begin{Highlighting}[]
\CommentTok{\# ─────────────────── Plot ─────────────}

\FunctionTok{ggplot}\NormalTok{() }\SpecialCharTok{+}
  \FunctionTok{geom\_rect}\NormalTok{(}
    \AttributeTok{data =}\NormalTok{ shade,}
    \FunctionTok{aes}\NormalTok{(}\AttributeTok{xmin =}\NormalTok{ xmin, }\AttributeTok{xmax =}\NormalTok{ xmax, }\AttributeTok{ymin =} \SpecialCharTok{{-}}\ConstantTok{Inf}\NormalTok{, }\AttributeTok{ymax =} \ConstantTok{Inf}\NormalTok{, }\AttributeTok{fill =}\NormalTok{ colour),}
    \AttributeTok{inherit.aes =} \ConstantTok{FALSE}\NormalTok{, }\AttributeTok{alpha =} \FloatTok{0.15}
\NormalTok{  ) }\SpecialCharTok{+}
  \FunctionTok{scale\_fill\_identity}\NormalTok{() }\SpecialCharTok{+}
  \FunctionTok{geom\_meatballs}\NormalTok{(}
    \AttributeTok{data =} \FunctionTok{tibble}\NormalTok{(}\AttributeTok{y =}\NormalTok{ nat\_weekly\_tfd),}
    \FunctionTok{aes}\NormalTok{(}\AttributeTok{y =}\NormalTok{ y),}
    \AttributeTok{colour =} \StringTok{"steelblue"}\NormalTok{, }\AttributeTok{size =} \DecValTok{2}\NormalTok{, }\AttributeTok{alpha =} \FloatTok{0.6}
\NormalTok{  ) }\SpecialCharTok{+}
  \FunctionTok{scale\_x\_continuous}\NormalTok{(}
    \AttributeTok{breaks =}\NormalTok{ year\_ticks}\SpecialCharTok{$}\NormalTok{day\_num,}
    \AttributeTok{labels =}\NormalTok{ year\_ticks}\SpecialCharTok{$}\NormalTok{year,}
    \AttributeTok{expand =} \FunctionTok{c}\NormalTok{(}\DecValTok{0}\NormalTok{, }\DecValTok{0}\NormalTok{)}
\NormalTok{  ) }\SpecialCharTok{+}
  \FunctionTok{labs}\NormalTok{(}
    \AttributeTok{x =} \StringTok{"Weeks starting in January 2017"}\NormalTok{,}
    \AttributeTok{y =} \StringTok{"Weekly all{-}cause deaths in the US (thousands)"}
\NormalTok{  ) }\SpecialCharTok{+}
  \FunctionTok{theme\_classic}\NormalTok{() }
\end{Highlighting}
\end{Shaded}

\pandocbounded{\includegraphics[keepaspectratio]{chapter-1_files/figure-latex/unnamed-chunk-2-1.pdf}}

\paragraph{Weekly US COVID-19 mortality
data}\label{weekly-us-covid-19-mortality-data}

\begin{Shaded}
\begin{Highlighting}[]
\CommentTok{\# ───Data ──────────────────}
\NormalTok{cv19\_deaths     }\OtherTok{\textless{}{-}}\NormalTok{ COVID19}\SpecialCharTok{$}\NormalTok{US\_weekly\_mort\_CV19}
\NormalTok{week\_diff    }\OtherTok{\textless{}{-}}\NormalTok{ COVID19}\SpecialCharTok{$}\NormalTok{US\_weekly\_excess\_mort\_2020}
\NormalTok{current\_date }\OtherTok{\textless{}{-}}\NormalTok{ COVID19}\SpecialCharTok{$}\NormalTok{US\_weekly\_excess\_mort\_2020\_dates}
\end{Highlighting}
\end{Shaded}

\begin{Shaded}
\begin{Highlighting}[]
\CommentTok{\# ─── Turn into tfd ──────────────────}

\NormalTok{death\_tf }\OtherTok{\textless{}{-}} \FunctionTok{tfd}\NormalTok{(}
  \FunctionTok{rbind}\NormalTok{(week\_diff, cv19\_deaths),}
  \AttributeTok{arg =} \FunctionTok{num\_grid}\NormalTok{(current\_date)}
\NormalTok{)}
\NormalTok{death\_tf}
\end{Highlighting}
\end{Shaded}

\begin{verbatim}
## tfd[2] on (0,357) based on 52 evaluations each
## interpolation by tf_approx_linear 
## week_diff: ( 0,1698);( 7,2222);(14,1011); ...
## cv19_deaths: ( 0,   0);( 7,   0);(14,   3); ...
\end{verbatim}

\begin{Shaded}
\begin{Highlighting}[]
\CommentTok{\# ── Prepare data for plotting ──────────────────}

\NormalTok{excess\_tibble }\OtherTok{\textless{}{-}} \FunctionTok{tibble}\NormalTok{(}
  \AttributeTok{death\_type =} \FunctionTok{c}\NormalTok{(}\StringTok{"All{-}cause excess"}\NormalTok{, }\StringTok{"COVID{-}19"}\NormalTok{),}
  \AttributeTok{curve      =}\NormalTok{ death\_tf}
\NormalTok{)}
\NormalTok{excess\_tibble}
\end{Highlighting}
\end{Shaded}

\begin{verbatim}
## # A tibble: 2 x 2
##   death_type                                         curve
##   <chr>                                          <tfd_reg>
## 1 All-cause excess [1]: ( 0,1698);( 7,2222);(14,1011); ...
## 2 COVID-19         [2]: ( 0,   0);( 7,   0);(14,   3); ...
\end{verbatim}

\begin{Shaded}
\begin{Highlighting}[]
\CommentTok{\# ──────────────── Plot ────────────────}
\DocumentationTok{\#\#Plot excess weekly US all{-}cause excess mortality and COVID{-}19 mortality in 2019}

\FunctionTok{ggplot}\NormalTok{(excess\_tibble, }\FunctionTok{aes}\NormalTok{(}\AttributeTok{y =}\NormalTok{ curve, }\AttributeTok{colour =}\NormalTok{ death\_type)) }\SpecialCharTok{+}
  \FunctionTok{geom\_meatballs}\NormalTok{(}\AttributeTok{size =} \DecValTok{3}\NormalTok{, }\AttributeTok{alpha =} \FloatTok{0.6}\NormalTok{) }\SpecialCharTok{+}
  \FunctionTok{scale\_colour\_manual}\NormalTok{(}\AttributeTok{values =} \FunctionTok{c}\NormalTok{(}\StringTok{"blue"}\NormalTok{, }\StringTok{"red"}\NormalTok{)) }\SpecialCharTok{+}
  \FunctionTok{scale\_x\_continuous}\NormalTok{(}
    \AttributeTok{name =} \StringTok{"Weeks starting January 2020"}\NormalTok{,}
    \AttributeTok{breaks =} \FunctionTok{as.numeric}\NormalTok{(}\FunctionTok{seq}\NormalTok{(reference\_date, }\FunctionTok{as.Date}\NormalTok{(}\StringTok{"2021{-}01{-}01"}\NormalTok{), }\AttributeTok{by =} \StringTok{"3 months"}\NormalTok{) }\SpecialCharTok{{-}}\NormalTok{ reference\_date),}
    \AttributeTok{labels =} \FunctionTok{format}\NormalTok{(}\FunctionTok{seq}\NormalTok{(reference\_date, }\FunctionTok{as.Date}\NormalTok{(}\StringTok{"2021{-}01{-}01"}\NormalTok{), }\AttributeTok{by =} \StringTok{"3 months"}\NormalTok{), }\StringTok{"\%b \%Y"}\NormalTok{)}
\NormalTok{  ) }\SpecialCharTok{+}
  \CommentTok{\# scale\_x\_continuous(}
  \CommentTok{\#   limits = c(0, max(num\_grid(current\_date))),  \# force axis to start at 0}
  \CommentTok{\#   breaks = num\_grid(current\_date)[seq(1, length(current\_date), 13)],}
  \CommentTok{\#   labels = format(current\_date[seq(1, length(current\_date), 13)], "\%b\textbackslash{}n\%Y"),}
  \CommentTok{\#   expand = c(0.01,0.1) \# This line removes the gap(not really)}
  \CommentTok{\# ) + \#coord\_cartesian(ylim = c(0, NA), expand = FALSE)+}
  \FunctionTok{labs}\NormalTok{(}
    \AttributeTok{x =} \StringTok{"Weeks starting in January 2020"}\NormalTok{,}
    \AttributeTok{y =} \StringTok{"All{-}cause excess and COVID{-}19 deaths in the US"}
\NormalTok{  ) }\SpecialCharTok{+}
  \FunctionTok{theme\_classic}\NormalTok{() }
\end{Highlighting}
\end{Shaded}

\pandocbounded{\includegraphics[keepaspectratio]{chapter-1_files/figure-latex/unnamed-chunk-6-1.pdf}}

\paragraph{Plot of all-cause cumulative excess
data}\label{plot-of-all-cause-cumulative-excess-data}

\begin{Shaded}
\begin{Highlighting}[]
\CommentTok{\# ── Data ───────────────────}
\NormalTok{new\_states        }\OtherTok{\textless{}{-}}\NormalTok{ COVID19}\SpecialCharTok{$}\NormalTok{US\_states\_names}
\NormalTok{state\_population    }\OtherTok{\textless{}{-}}\NormalTok{ COVID19}\SpecialCharTok{$}\NormalTok{US\_states\_population}
\NormalTok{states\_excess }\OtherTok{\textless{}{-}}\NormalTok{ COVID19}\SpecialCharTok{$}\NormalTok{States\_excess\_mortality}
\end{Highlighting}
\end{Shaded}

\begin{Shaded}
\begin{Highlighting}[]
\CommentTok{\# ── Data Transformation ───────────────────}

\CommentTok{\#Normalize to rate per 1 million people}
\NormalTok{rate\_mat }\OtherTok{\textless{}{-}} \FunctionTok{sweep}\NormalTok{(states\_excess, }\DecValTok{1}\NormalTok{, state\_population }\SpecialCharTok{/} \FloatTok{1e6}\NormalTok{, }\StringTok{"/"}\NormalTok{)}

\CommentTok{\#row by row , and replace each row with its cumulative sum.}
\CommentTok{\#We transpose it back so that rows = states, columns = weeks again.}
\NormalTok{cum\_mat  }\OtherTok{\textless{}{-}} \FunctionTok{t}\NormalTok{(}\FunctionTok{apply}\NormalTok{(rate\_mat, }\DecValTok{1}\NormalTok{, cumsum))}
\end{Highlighting}
\end{Shaded}

\begin{Shaded}
\begin{Highlighting}[]
\CommentTok{\# ────────────── Turn into tfd ──────────────────}

\CommentTok{\#So now we end up with a tfd object where each row(curve represents a state)}
\NormalTok{state\_tf }\OtherTok{\textless{}{-}} \FunctionTok{tfd}\NormalTok{(cum\_mat, }\AttributeTok{arg =} \FunctionTok{num\_grid}\NormalTok{(current\_date))}

\NormalTok{state\_tf}
\end{Highlighting}
\end{Shaded}

\begin{verbatim}
## tfd[52] on (0,357) based on 52 evaluations each
## interpolation by tf_approx_linear 
## [1]: ( 0, 0.8);( 7, 8.3);(14,-6.9); ...
## [2]: ( 0,  -5);( 7,  19);(14,  26); ...
## [3]: ( 0,  -7);( 7,  -1);(14,   5); ...
## [4]: ( 0,  -7);( 7,  21);(14,  20); ...
## [5]: ( 0,   5);( 7,   7);(14,  12); ...
##     [....]   (47 not shown)
\end{verbatim}

\begin{Shaded}
\begin{Highlighting}[]
\CommentTok{\# ── Prepare data for plotting ──────────────────}

\NormalTok{state\_df }\OtherTok{\textless{}{-}} \FunctionTok{tibble}\NormalTok{(}
  \AttributeTok{state =}\NormalTok{ new\_states,}
  \AttributeTok{curve =}\NormalTok{ state\_tf}
\NormalTok{)}
\NormalTok{state\_df}
\end{Highlighting}
\end{Shaded}

\begin{verbatim}
## # A tibble: 52 x 2
##    state                                                   curve
##    <chr>                                               <tfd_reg>
##  1 Alabama               [1]: ( 0, 0.8);( 7, 8.3);(14,-6.9); ...
##  2 Alaska                [2]: ( 0,  -5);( 7,  19);(14,  26); ...
##  3 Arizona               [3]: ( 0,  -7);( 7,  -1);(14,   5); ...
##  4 Arkansas              [4]: ( 0,  -7);( 7,  21);(14,  20); ...
##  5 California            [5]: ( 0,   5);( 7,   7);(14,  12); ...
##  6 Colorado              [6]: ( 0,   3);( 7,  10);(14,  14); ...
##  7 Connecticut           [7]: ( 0, 0.3);( 7, 1.1);(14, 4.2); ...
##  8 Delaware              [8]: ( 0,  14);( 7,  13);(14,   4); ...
##  9 District of Columbia  [9]: ( 0,   3);( 7,  11);(14,  60); ...
## 10 Florida              [10]: ( 0,   6);( 7,  15);(14,  24); ...
## # i 42 more rows
\end{verbatim}

\begin{Shaded}
\begin{Highlighting}[]
\CommentTok{\# ── Plot ────────────────}


\NormalTok{emph\_states }\OtherTok{\textless{}{-}} \FunctionTok{c}\NormalTok{(}\StringTok{"New Jersey"}\NormalTok{, }\StringTok{"Louisiana"}\NormalTok{, }\StringTok{"California"}\NormalTok{, }\StringTok{"Maryland"}\NormalTok{, }\StringTok{"Texas"}\NormalTok{)}
\NormalTok{emph\_cols   }\OtherTok{\textless{}{-}} \FunctionTok{c}\NormalTok{(}\StringTok{"darkseagreen3"}\NormalTok{, }\StringTok{"red"}\NormalTok{, }\StringTok{"plum3"}\NormalTok{, }\StringTok{"deepskyblue4"}\NormalTok{, }\StringTok{"salmon"}\NormalTok{)}

\FunctionTok{ggplot}\NormalTok{() }\SpecialCharTok{+}
  \FunctionTok{geom\_spaghetti}\NormalTok{(}
    \AttributeTok{data   =}\NormalTok{ state\_df,}
    \FunctionTok{aes}\NormalTok{(}\AttributeTok{y =}\NormalTok{ curve),}
    \AttributeTok{colour =} \StringTok{"grey20"}\NormalTok{, }\AttributeTok{alpha =}\NormalTok{ .}\DecValTok{15}
\NormalTok{  ) }\SpecialCharTok{+}
  \FunctionTok{geom\_spaghetti}\NormalTok{(}
    \AttributeTok{data   =} \FunctionTok{filter}\NormalTok{(state\_df, state }\SpecialCharTok{\%in\%}\NormalTok{ emph\_states),}
    \FunctionTok{aes}\NormalTok{(}\AttributeTok{y =}\NormalTok{ curve, }\AttributeTok{colour =}\NormalTok{ state), }\AttributeTok{alpha =} \DecValTok{5}
\NormalTok{  ) }\SpecialCharTok{+}
  \FunctionTok{scale\_colour\_manual}\NormalTok{(}\AttributeTok{values =} \FunctionTok{setNames}\NormalTok{(emph\_cols, emph\_states)) }\SpecialCharTok{+}
  \FunctionTok{scale\_x\_continuous}\NormalTok{(}
    \AttributeTok{name =} \StringTok{"Weeks starting January 2020"}\NormalTok{,}
    \AttributeTok{breaks =} \FunctionTok{as.numeric}\NormalTok{(}\FunctionTok{seq}\NormalTok{(reference\_date, }\FunctionTok{as.Date}\NormalTok{(}\StringTok{"2021{-}01{-}01"}\NormalTok{), }\AttributeTok{by =} \StringTok{"3 months"}\NormalTok{) }\SpecialCharTok{{-}}\NormalTok{ reference\_date),}
    \AttributeTok{labels =} \FunctionTok{format}\NormalTok{(}\FunctionTok{seq}\NormalTok{(reference\_date, }\FunctionTok{as.Date}\NormalTok{(}\StringTok{"2021{-}01{-}01"}\NormalTok{), }\AttributeTok{by =} \StringTok{"3 months"}\NormalTok{), }\StringTok{"\%b \%Y"}\NormalTok{)}
\NormalTok{  ) }\SpecialCharTok{+}
  \CommentTok{\# scale\_x\_continuous(}
  \CommentTok{\#   breaks = num\_grid(current\_date)[seq(1, length(current\_date), 13)],}
  \CommentTok{\#   labels = format(current\_date[seq(1, length(current\_date), 13)], "\%b\textbackslash{}n\%Y"),}
  \CommentTok{\#   expand = c(0, 0)}
  \CommentTok{\# ) +}
  \FunctionTok{coord\_cartesian}\NormalTok{(}\AttributeTok{ylim =} \FunctionTok{c}\NormalTok{(}\SpecialCharTok{{-}}\DecValTok{50}\NormalTok{, }\DecValTok{2500}\NormalTok{)) }\SpecialCharTok{+}
  \FunctionTok{labs}\NormalTok{(}
    \AttributeTok{x =} \StringTok{"Weeks starting January 2020"}\NormalTok{,}
    \AttributeTok{y =} \StringTok{"US states cumulative excess deaths per million"}
\NormalTok{  ) }\SpecialCharTok{+}
  \FunctionTok{theme\_classic}\NormalTok{() }\SpecialCharTok{+}
  \FunctionTok{theme}\NormalTok{(}\AttributeTok{legend.position =} \StringTok{"right"}\NormalTok{)}
\end{Highlighting}
\end{Shaded}

\pandocbounded{\includegraphics[keepaspectratio]{chapter-1_files/figure-latex/unnamed-chunk-11-1.pdf}}

\paragraph{Plot of COVID-19 cumulative
data}\label{plot-of-covid-19-cumulative-data}

\begin{Shaded}
\begin{Highlighting}[]
\CommentTok{\# ── Data ────────────}
\NormalTok{states\_CV19\_mort }\OtherTok{\textless{}{-}}\NormalTok{ COVID19}\SpecialCharTok{$}\NormalTok{States\_CV19\_mortality}
\end{Highlighting}
\end{Shaded}

\begin{Shaded}
\begin{Highlighting}[]
\CommentTok{\# ── Data Transformation ───────────────────}
\NormalTok{per\_mil\_mat  }\OtherTok{\textless{}{-}} \FunctionTok{sweep}\NormalTok{(states\_CV19\_mort, }\DecValTok{1}\NormalTok{, state\_population }\SpecialCharTok{/} \FloatTok{1e6}\NormalTok{, }\StringTok{"/"}\NormalTok{)}
\NormalTok{cum\_covid    }\OtherTok{\textless{}{-}} \FunctionTok{t}\NormalTok{(}\FunctionTok{apply}\NormalTok{(per\_mil\_mat, }\DecValTok{1}\NormalTok{, }\ControlFlowTok{function}\NormalTok{(x) }\FunctionTok{cumsum}\NormalTok{(}\FunctionTok{replace\_na}\NormalTok{(x, }\DecValTok{0}\NormalTok{))))}
\end{Highlighting}
\end{Shaded}

\begin{Shaded}
\begin{Highlighting}[]
\CommentTok{\# ────────────── Turn into tfd ──────────────────}

\NormalTok{covid\_tf }\OtherTok{\textless{}{-}} \FunctionTok{tfd}\NormalTok{(cum\_covid, }\AttributeTok{arg =} \FunctionTok{num\_grid}\NormalTok{(current\_date))}
\NormalTok{covid\_tf}
\end{Highlighting}
\end{Shaded}

\begin{verbatim}
## tfd[52] on (0,357) based on 52 evaluations each
## interpolation by tf_approx_linear 
## [1]: ( 0,0);( 7,0);(14,0); ...
## [2]: ( 0,0);( 7,0);(14,0); ...
## [3]: ( 0,0);( 7,0);(14,0); ...
## [4]: ( 0,0);( 7,0);(14,0); ...
## [5]: ( 0,0);( 7,0);(14,0); ...
##     [....]   (47 not shown)
\end{verbatim}

\begin{Shaded}
\begin{Highlighting}[]
\CommentTok{\# ── Data Transformation ───────────────────}
\NormalTok{covid\_df }\OtherTok{\textless{}{-}} \FunctionTok{tibble}\NormalTok{(}
  \AttributeTok{state =}\NormalTok{ new\_states,}
  \AttributeTok{curve =}\NormalTok{ covid\_tf}
\NormalTok{)}
\NormalTok{covid\_df}
\end{Highlighting}
\end{Shaded}

\begin{verbatim}
## # A tibble: 52 x 2
##    state                                          curve
##    <chr>                                      <tfd_reg>
##  1 Alabama               [1]: ( 0,0);( 7,0);(14,0); ...
##  2 Alaska                [2]: ( 0,0);( 7,0);(14,0); ...
##  3 Arizona               [3]: ( 0,0);( 7,0);(14,0); ...
##  4 Arkansas              [4]: ( 0,0);( 7,0);(14,0); ...
##  5 California            [5]: ( 0,0);( 7,0);(14,0); ...
##  6 Colorado              [6]: ( 0,0);( 7,0);(14,0); ...
##  7 Connecticut           [7]: ( 0,0);( 7,0);(14,0); ...
##  8 Delaware              [8]: ( 0,0);( 7,0);(14,0); ...
##  9 District of Columbia  [9]: ( 0,0);( 7,0);(14,0); ...
## 10 Florida              [10]: ( 0,0);( 7,0);(14,0); ...
## # i 42 more rows
\end{verbatim}

\begin{Shaded}
\begin{Highlighting}[]
\CommentTok{\# ──────────────── Plot ────────────────}
\CommentTok{\#Plot of COVID{-}19 cumulative data}

\FunctionTok{ggplot}\NormalTok{() }\SpecialCharTok{+}
  \FunctionTok{geom\_spaghetti}\NormalTok{(}
    \AttributeTok{data   =}\NormalTok{ covid\_df,}
    \FunctionTok{aes}\NormalTok{(}\AttributeTok{y =}\NormalTok{ curve),}
    \AttributeTok{colour =} \StringTok{"black"}\NormalTok{, }\AttributeTok{alpha =} \FloatTok{0.1}
\NormalTok{  ) }\SpecialCharTok{+}
  \CommentTok{\# highlighted states}
  \FunctionTok{geom\_spaghetti}\NormalTok{(}
    \AttributeTok{data   =} \FunctionTok{filter}\NormalTok{(covid\_df, state }\SpecialCharTok{\%in\%}\NormalTok{ emph\_states),}
    \FunctionTok{aes}\NormalTok{(}\AttributeTok{y =}\NormalTok{ curve, }\AttributeTok{colour =}\NormalTok{ state), }\AttributeTok{alpha =} \DecValTok{2}
\NormalTok{  ) }\SpecialCharTok{+}
  \FunctionTok{scale\_colour\_manual}\NormalTok{(}\AttributeTok{values =} \FunctionTok{setNames}\NormalTok{(emph\_cols, emph\_states)) }\SpecialCharTok{+}
  \FunctionTok{scale\_x\_continuous}\NormalTok{(}
    \AttributeTok{name =} \StringTok{"Weeks starting January 2020"}\NormalTok{,}
    \AttributeTok{breaks =} \FunctionTok{as.numeric}\NormalTok{(}\FunctionTok{seq}\NormalTok{(reference\_date, }\FunctionTok{as.Date}\NormalTok{(}\StringTok{"2021{-}01{-}01"}\NormalTok{), }\AttributeTok{by =} \StringTok{"3 months"}\NormalTok{) }\SpecialCharTok{{-}}\NormalTok{ reference\_date),}
    \AttributeTok{labels =} \FunctionTok{format}\NormalTok{(}\FunctionTok{seq}\NormalTok{(reference\_date, }\FunctionTok{as.Date}\NormalTok{(}\StringTok{"2021{-}01{-}01"}\NormalTok{), }\AttributeTok{by =} \StringTok{"3 months"}\NormalTok{), }\StringTok{"\%b \%Y"}\NormalTok{)}
\NormalTok{  ) }\SpecialCharTok{+}
  \CommentTok{\# scale\_x\_continuous(}
  \CommentTok{\#   breaks = num\_grid(current\_date)[seq(1, length(current\_date), 13)],}
  \CommentTok{\#   labels = format(current\_date[seq(1, length(current\_date), 13)], "\%b\textbackslash{}n\%Y"),}
  \CommentTok{\#   expand = c(0, 0)}
  \CommentTok{\# ) +}
  \FunctionTok{coord\_cartesian}\NormalTok{(}\AttributeTok{ylim =} \FunctionTok{c}\NormalTok{(}\DecValTok{0}\NormalTok{, }\ConstantTok{NA}\NormalTok{)) }\SpecialCharTok{+}
  \FunctionTok{labs}\NormalTok{(}
    \AttributeTok{x   =} \StringTok{"Weeks starting January 2020"}\NormalTok{,}
    \AttributeTok{y   =} \StringTok{"Cumulative COVID{-}19 deaths per million (US states)"}\NormalTok{,}
    \AttributeTok{col =} \ConstantTok{NULL}
\NormalTok{  ) }\SpecialCharTok{+}
  \FunctionTok{theme\_classic}\NormalTok{() }\SpecialCharTok{+}
  \FunctionTok{theme}\NormalTok{(}\AttributeTok{legend.position =} \StringTok{"right"}\NormalTok{)}
\end{Highlighting}
\end{Shaded}

\pandocbounded{\includegraphics[keepaspectratio]{chapter-1_files/figure-latex/unnamed-chunk-16-1.pdf}}

\end{document}
